\documentclass{article}
\usepackage[utf8]{inputenc}
\usepackage[italian]{babel}
\usepackage[T1]{fontenc}
\usepackage[italian]{varioref}
\usepackage{datetime}
\selectlanguage{italian}
\usepackage[Algoritmo]{algorithm}
\usepackage{algpseudocode}

\hfuzz=100.0pt  % ignore paragraph lengths warnings

\usepackage[square,sort,comma,numbers]{natbib}

\newdate{date}{06}{09}{2021}
\title{\textsc{LittleMan}\\
%\large Relazione del progetto per l'insegnamento di Algoritmi e Strutture di Dati}
\large Relazione del progetto per il corso di Algoritmi e Strutture di Dati}
\author{
  Mattia Girolimetto (0000977478),
  Luca Tagliavini (0000971133)
}
\date{
	Universit\`a di Bologna \\
  \displaydate{date}
}

\begin{document}

\maketitle

\section*{Problema computazionale}

Lo scopo del progetto \`e quello di implementare un algoritmo efficiente e ottimale
volto alla ricerca delle mosse migliori in un gioco $(m,n,k)$, dove si devono
allineare $k$ simboli in una griglia $m \times n$.

Supponendo non esistano stati invalidi il nodo iniziale (la griglia vuota)
ha $m \times n$ figli, i quali avranno a loro volta $(m \times n)-1$ figli e cos\`i
via fino a profondit\`a $mn$ dove ci saranno tutte le foglie. Il numero totale
dei nodi \`e dunque superiormente limitato da $O((m \times n) \cdot ((m \times n)-1)
\cdot \ldots \cdot 1) = O((m \times n)!)$.
Il fattore di diramazione \`e dunque molto alto ad ogni stato della partita,
il che impone a un qualunque algoritmo di ricerca per forza bruta la
limitazione in profondit\`a su griglie non banali. 

\section*{Soluzione adottata}

La soluzione fa uso di una variante dell'algoritmo \textsc{MiniMax} con potatura
\textsc{AlphaBeta} denominata \textsc{PrincipalVariationSearch}~\cite{negascout}. Questa
consiste in una ricerca limitata in profondit\`a analoga ad \textsc{AlphaBeta}, espandendo 
interamente i nodi pi\`u promettenti e parzialmente quelli restanti. Applicando
un' \textsc{IterativeDeepeningSearch}~\cite{id} si possono ordinare i sottoalberi
basandosi sui valori euristici delle ricerche precedenti e si pu\`o raggiungere
la profondit\`a massima nei limiti imposti.

Comuputando l'albero si nota facilmente che diverse combinazioni di
mosse possono portare alla stessa situazione di gioco.
Per evitare di analizzarli pi\`u volte, gli stati di gioco vengono quindi mantenuti 
dentro una \emph{tabella delle trasposizioni}. Il valore di quelli non finali
%Diverse combinazioni di mosse possono portare a stati gi\`a analizzati
%precedentemente, i quali vengono mantenuti dentro una tabella delle trasposizioni
%per evitare di valutarli pi\`u volte. Il valore degli stati di gioco non finali
viene stimato da una componente euristica che tiene in considerazione il numero
%di serie di ogni giocatore, la relativa lunghezza e favorisce quelle di lunghezza
di serie di ogni giocatore e la relativa lunghezza favorendo quelle di lunghezza
$k-1$, $k-2$ e $k-3$.

\subsection*{Iterative Deepening}

La ricerca della mossa migliore viene gestita da un algoritmo di
\textsc{IterativeDeepeningSearch}~\cite{id}, il quale procede applicando MiniMax % non pvs?
%con profondit\`a sempre maggiore e restituisce il risultato pi\`u recente allo
con profondit\`a sempre maggiore e restituisce l'ultimo risultato trovato prima dello
scadere del tempo. Questo non peggiora la complessit\`a asintotica in quanto il
solo costo della ricerca alla massima profondit\`a assorbe quello di tutte
le chiamate precedenti. Si ha tuttavia un aumento delle operazioni totali svolte
dal calcolatore, mitigato per\`o dall'utilizzo della \textsc{PrincipalVariationSearch} 
e dalle potature derivate dall'ordinamento dei nodi ottenuto grazie a questa tecnica.

\subsection*{Tabella delle trasposizioni}

La cache fa uso della struttura dati HashMap fornita da Java generando le chiavi 
in modo incrementale tramite la tecnica di \emph{Zobrist}~\cite{zobrist}. 
%Ad ogni chiave sono associate le seguenti informazioni:
Ad ogniuna di esse sono associate le seguenti informazioni:
\begin{enumerate}
  \item Numero di celle marcate e ultimo simbolo giocato
  \item Profondit\`a della ricerca
  \item Tipo e valore della griglia
\end{enumerate}
Poich\`e una board $(m,n)$ pu\`o al pi\`u assumere $3^{mn}$ stati differenti e
un \verb!long! in Java pu\`o contenere al pi\`u $2^{64} \approx 1.8 \cdot 10^{19}$ valori distinti,
esistono configurazioni che creano collisioni: ad esempio un gioco $(8, 8, 4)$
pu\`o avere fino a $3^{64} \approx 2.3 \cdot 10^{30}$ stati. Per limitare questo
fenomeno si verifica che il numero di mosse e l'ultimo simbolo giocato combacino. % come si fa a confrontare un numero con un simbolo?

Per evitare ulteriori collisioni interne alla HashMap, alla fine di
% ogni turno, la tabella delle trasposizioni viene privata delle griglie \emph{
ogni turno, vengono rimosse dalla tabella delle trasposizioni tutte le griglie \emph{
inaccessibili}. Una griglia viene definita inaccessibile quando ha un numero di
celle marcate inferiore a quello del tavolo di gioco attuale incrementato di $1$.

\subsection*{Principal Variation Search}

L'algoritmo visita in modo ordinato i figli di ogni stato di gioco applicando
limiti $\alpha$-$\beta$ differenti ad ognuno di essi al fine di aumentare il numero di potature.
Si analizza in modo approfondito il nodo pi\`u promettente, denominato \emph{Principal
Variation}, per poi visitare i restanti nodi con una ricerca a finestra \emph{nulla}~\cite{scout}
(dove i due limiti differiscono di $1$). Se questa restituisce un valore compreso
tra gli $\alpha$ e $\beta$ originali potrebbe essere d'interesse dunque viene
% negli $\alpha$-$\beta$ originali potrebbe essere d'interesse e dunque viene
svolta una ricerca classica.
% TODO IN REALTA' E' EURISTICA QUESTA E FAREBBE BENE METTERLA SOTTO

Nel caso in cui esista una serie $k-1$ per un qualunque giocatore si provvede a
riempire la cella mancante, evitando la ricerca di forza bruta. Viene data la
precedenza alle serie appartenenti al giocatore che deve svolgere la mossa.

\subsection*{Valutazione euristica}

La griglia di gioco mantiene al suo interno una valutazione euristica 
aggiornata progressivamente con l'avanzare della partita (Algoritmo \vref{alg:eval}).
Quando una cella viene marcata si valuta la variazione del valore euristico osservando
le righe, colonne e diagonali passanti per essa. Vengono favorite le serie consecutive 
di $k-1, k-2$ o $k-3$ elementi, dando un peso inferiore a segmenti pi\`u corti.
La quotazione di un tavolo di gioco \`e il risultato della differenza tra i
%valori delle serie dei due giocatori. Oltretutto vengono favorite griglie che
%assumono valori promettenti con il minor numero di mosse, dividendo ogni
%valutazione per la profondit\`a dello stato di gioco.
valori delle serie dei due giocatori. Inoltre, per favorire le griglie che
assumono valori promettenti con il minor numero di mosse, ogni
valutazione viene divisa per la profondit\`a dello stato di gioco.

Questo approccio ricorda quello scelto da Chuan per il gioco di Go~\cite{chuan}, % Chuan è il nome completo?
% tuttavia si \`e preferito favorire anche serie di lunghezza inferiore ai fini di
% stimare una pi\`u accurata \emph{Principal Variation}, sopratutto durante le  - Soprattutto va con 4 't'!
tuttavia si \`e preferito favorire anche serie di lunghezza inferiore per migliorare 
l'accuratezza della \emph{Principal Variation}, soprattutto durante le mosse iniziali.

\begin{algorithm}[H]
  \caption{Valutazione eurstica procedurale}
  \label{alg:eval}
  \begin{algorithmic}[0]
    \Procedure {Eval}{$i$, $j$}
      \State $value \gets 0$
      \For{$(ii, jj) \in \Call{Row}{i}$}
        \State $value \gets value + \Call{CellValue}{ii, jj, 0, 1}$
      \EndFor
      \For{$(ii, jj) \in \Call{Column}{j}$}
        \State $value \gets value + \Call{CellValue}{ii, jj, 1, 0}$
      \EndFor
      \For{$(ii, jj) \in \Call{Diagonal}{i, j}$}
        \State $value \gets value + \Call{CellValue}{ii, jj, 1, 1}$
      \EndFor
      \For{$(ii, jj) \in \Call{AntiDiagonal}{i, j}$}
        \State $value \gets value + \Call{CellValue}{ii, jj, 1, -1}$
      \EndFor
      \State \Return $value$
    \EndProcedure
    \Statex
    \Procedure {CellValue}{$i$, $j$, $\delta_i$, $\delta_j$}
      \If{$n_{free} + n_1 + n_2 \geq k$} \Comment{Se la serie \`e troppo lunga}
        \State $s \gets B[i - \delta_i \cdot k][j - \delta_j \cdot k]$ \Comment{Stato della prima cella nella serie}
        \State decrement $n_{free}, n_1, n_2$ by $1$ based on $s$
      \EndIf
      \State increment $n_{free}, n_1, n_2$ by $1$ based on $B[i][j]$
      \Statex
      \If{$n_1 + p_{free} = k$}
        \Comment{Si restituisce la valutazione della serie}
        \State \Return $color \cdot (\Call{LargeSeriesConstant}{n_{free}} + n_1^2)$
      \ElsIf{$n_2 + p_{free} = k$}
        \State \Return $-color \cdot (\Call{LargeSeriesConstant}{n_{free}} + n_2^2)$
      \Else
        \State \Return $0$
      \EndIf
    \EndProcedure
  \end{algorithmic}
\end{algorithm}

\subsection*{Altre euristiche considerate}

% In questa sezione saranno presentati approcci che sono stati considerati,
In questa sezione sono presentati approcci che sono stati considerati e
implementati, ma in definitiva scartati in quanto non hanno offerto un
miglioramento apprezzabile allo stile di gioco dell'algoritmo.

\subsubsection*{Valutazione euristica degli spazi vuoti}

Nel gioco generalizzato del $(m, n, k)$, specialmente per valori di $k < \max\{m,n\}$,
una sequenza con celle libere ai lati pu\`o rivelarsi pi\`u vantaggiosa di una
% limitata: quando la sequenza \`e del tipo $k-1$ ed ha entrambi i lati liberi
% essa garantisce la vittoria al giocatore. Si pu\`o dunque estendere l'Algoritmo
limitata. Ad esempio quando una sequenza \`e del tipo $k-1$ ed ha entrambi i lati liberi
garantisce la vittoria al giocatore. Si pu\`o dunque estendere l'Algoritmo
%\vref{alg:eval} affinch\`e sommi un bonus se la serie sotto analisi contiene celle
% libere attorno a se. Ecco una possibile implementazione che si pu\`o trovare
\vref{alg:eval} affinch\`e sommi un bonus al valore di questo tipo di serie. 
Ecco una possibile implementazione che si pu\`o trovare
commentata anche nel codice sorgente:

\begin{algorithm}[H]
  \caption{Valutazione delle serie favorendo spazi liberi adiacenti}
  \label{alg:eval_free}
  \begin{algorithmic}[0]
    \Procedure {CellValue}{$i$, $j$, $\delta_i$, $\delta_j$}
    \Comment{La logica rimane invariata fino alla valutazione della serie attuale}
    \State $bonus \gets 0$
    \If{$B[i + \delta_i][j + \delta_j] = free$}
      \State $bonus \gets bonus + \textsc{BonusConstant}$
    \EndIf
    \If{$B[i - \delta_i \cdot (k+1)][j - \delta_j \cdot (k+1)] = free$}
      \State $bonus \gets bonus + \textsc{BonusConstant}$
    \EndIf
    \Statex
    \If{$n_1 + p_{free} = k$}
      \State \Return $color \cdot (\Call{LargeSeriesConstant}{n_{free}} + n_1^2 + bonus)$
    \ElsIf{$n_2 + p_{free} = k$}
    \State \Return $-color \cdot (\Call{LargeSeriesConstant}{n_{free}} + n_2^2 + bonus)$
    \Else
      \State \Return $0$
    \EndIf
    \EndProcedure
  \end{algorithmic}
\end{algorithm}

Va ricordato che nell'Algoritmo~\ref{alg:eval_free} il controllo della validit\`a delle
posizioni $(i + \delta_i, j + \delta_j)$ e $(i - \delta_i \cdot (k+1), j - \delta_j \cdot (k+1))$
\`e stato omesso per brevit\`a e chiarezza del codice.

\subsubsection*{Ricerca quiescente}

Un problema ben noto negli algoritmi di ricerca a profondit\`a limitata \`e
%quello dell'\emph{effetto orizzonte}, causato dall'impossibilt\`a della procedura
%di visitare l'albero oltre un certo margine. Ci\`o porta spesso alla scelta di   
quello dell'\emph{effetto orizzonte}, causato dall'impossibilt\`a 
di visitare l'albero di gioco oltre un certo livello. Ci\`o porta spesso l'algoritmo a scegliere 
mosse che si rivelano catastrofiche in pochi turni di gioco. 
%Una soluzione al problema chiamata \emph{ricerca quiescente} \cite{quiescence} fu per la prima
%volta proposta  da \citeauthor{quiescence} nel \citeyear{quiescence}.
\citeauthor{quiescence} nel \citeyear{quiescence} per primo propose una soluzione 
al problema: la \emph{ricerca quiescente}. % oppure: al problema chiamata \emph{...} 


L'idea \`e quella di esplorare ulteriormente gli stati che appaiono "calmi", o  % spiegare come capisco se un nodo è calmo
appunto "quiescenti", poich\`e sono quelli pi\`u proni a degenerare in sconfitte.
Si pu\`o dunque modificare la funzione di valutazione in modo da applicare una
\textsc{QuiescensceSearch} come mostrato di seguito:

\begin{algorithm}[H]
  \caption{Ricerca quiescente con struttura \textsc{NegaMax}}
  \label{qs}
  \begin{algorithmic}[0]
    \Procedure{Evalaute}{$board$, $raw$}
      \State $eval \gets $ valutazione euristica
      \If{$raw$ \textbf{or} $|eval| > \textsc{QuiescenceThreshold}$}
        \State \Return $eval$
      \Else
        \State $color \gets 1$
        \If{next moving player of the $board$ is $enemy$}
          \State $color \gets -1$
        \EndIf
        \State \Return \Call{QuiescenceSearch}{$board$, $color$, \textsc{QuiescenceDepth}}
      \EndIf
    \EndProcedure
    \Statex
    \Procedure{QuiescenceSearch}{$node$, $color$, $depth$}
    \If{$depth = 0$ \textbf{or} node is leaf}
      \State \Return \Call{Evaluate}{$node$, $true$}
    \EndIf
    \State $best \gets -\infty$
    \For{$child$ of $node$}
      \State $best \gets \max\{best, \Call{QuiescenceSearch}{child, -color, depth-1}\}$
    \EndFor
    \State \Return $-best$
    \EndProcedure
  \end{algorithmic}
\end{algorithm}

% spostare e riformulare queste 3 righe sopra allo pseudocodice. Vedi commento sopra
In questo modo un ristretto gruppo di nodi selezionati (coloro che hanno valore
euristico inferiore a \textsc{QuiescenceThreshold}) vengono visitati a maggiore
profondit\`a nella speranza di stabilizzare la loro valutazione.

Nella pratica questa modifica non ha portato a significativi miglioramenti
ma ha caratterizzato l'AI con uno stile di gioco pi\`u difensivo, il quale ne ha
peggiorato le prestazioni contro giocatori poco astuti o casuali.
Si pu\`o imputare questo fallimento a due fattori:
\begin{enumerate}
  \item La scarsa precisione della valutazione euristica.
  \item Le propriet\`a dei giochi $(m,n,k)$ per cui qualunque mossa di un
    giocatore migliora il proprio punteggio: in questo modo i valori delle
    valutazioni oscillano e non convergono mai ad una stima appropriata, in
    particolar modo a basse profondit\`a.
\end{enumerate}

\subsection*{Miglioramenti futuri}

Sono noti in letteratura una serie di algoritmi e valutazioni euristiche utili
nel gioco $(m, n, k)$ che potrebbero migliorare le prestazioni 
%nel gioco $(m, n, k)$ che potrebbero portare significativi miglioramenti alle performance
attuali del giocatore. Ecco un elenco dei pi\`u significativi:

\begin{enumerate}
  \item Adottare un' euristica per la valutazione pi\`u raffinata. Questa avrebbe 
  %\item Adottare una euristica per la valutazione pi\`u raffinata avrebbe
    probabilmente il maggiore impatto: una valutazione delle minacce~\cite{heur}
    analoga a quella proposta da \citeauthor{heur} migliorerebbe significativamente
    la precisione sugli stati pi\`u avanzati del gioco. \label{newheur}
  \item Adottare un' euristica in grado di tenere traccia delle minacce. Con una come quella
  %\item Con una euristica in grado di tenere traccia delle minacce, come quella
    descritta nel punto \vref{newheur}, potrebbe rivelarsi importante l'utilizzo
    della \textsc{QuiescenceSearch}.
  \item \`E stato infine mostrato come altre variazioni di \textsc{AlphaBeta} e
    \textsc{IterativeDeepeningSearch} come MTD(f)~\cite{mtdf} possono portare a
    risultati ancora migliori nel gioco degli scacchi. Riteniamo sia prima
    necessario compiere il passo \vref{newheur} affinch\`e si riveli efficace anche nel nostro ambito.
\end{enumerate}

% \pagebreak
\bibliography{report}
\bibliographystyle{IEEEtranN}

\end{document}
