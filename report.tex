\documentclass{article}
\usepackage[utf8]{inputenc}
\usepackage[italian]{babel}
\usepackage[T1]{fontenc}

\title{Galileo Galilei \\
\large Relazione del progetto per l'insegnamento di Algoritmi e Strutture di Dati}
\author{
  Mattia Girolimetto (0000977478),
  Luca Tagliavini (0000971133)
}
\date{
	Universit\`a di Bologna \\
  \today
}

\begin{document}

\maketitle

\section{Problema computazionale}

Lo scopo del progetto \`e quello di implementare un algoritmo efficiente e ottimale
volto alla ricerca delle mosse migliori in un gioco $(m,n,k)$, dove si devono
allineare $k$ simboli in una griglia $m \times n$.

Si fa uso di una variante dell'algoritmo MiniMax con potatura
AlphaBeta denominata \emph{Principal Variation Search}~\cite{negascout}. Questa
prevede  una ricerca limitata in profondit\`a analoga a minimax, espandendo 
interamente i nodi pi\`u promettenti e parzialmente quelli restanti. Applicando
un \emph{Iterative Deepening}~\cite{id} si possono ordinare i sottoalberi
basandosi sui valori euristici delle ricerche precedenti e si pu\`o raggiungere
la profondit\`a massima nei limiti imposti.

Diverse combinazioni di mosse possono portare a stati gi\`a analizzati
precedentemente, i quali vengono mantenuti dentro una cache per evitare di
valutarli pi\`u volte. Il valore degli stati di gioco non finali viene stimato
da una componente euristica che tiene in considerazione il numero di serie di
ogni giocatore assieme alla relativa lunghezza e alla vicinanza ad altre celle
libere.

\section{Scelte Progettuali}

\subsection{Select Cell e Iterative Deepening}
Dopo aver aggiornato la copia locale della griglia di gioco si cerca la cella
migliore da giocare sfruttando Iterative Deepening. 
In questo modo si eseguono ricerche MiniMax con profondit\`a sempre maggiore
e si restituisce il risultato pi\`u recente. Questo non
peggiora la complessit\`a in quanto il solo costo asintotico di MiniMax alla
massima profondit\`a assorbe quello di tutte le ricerche precedenti.
Tuttavia aumenta sicuramente il numero di operazioni svolte dal calcolatore.
Per limitare questo danno si ricorre alla Principal Variation Search.

\subsection{Cache}
La cache fa uso della struttura dati HashMap fornita da Java generando le chiavi 
in modo incrementale tramite la tecnica di \emph{Zobrist}~\cite{zobrist}. A ogni
chiave sono associate le seguenti informazioni:
\begin{enumerate}
  \item Numero di mosse e ultimo simbolo giocato
  \item Profondit\`a della ricerca
  \item Valore della griglia
\end{enumerate}
Poich\`e una board $(m,n)$ pu\`o al pi\`u assumere $3^{mn}$ stati differenti e
un \verb!long! pu\`o contenere al pi\`u $2^{64} \approx 1.8 \cdot 10^{19}$,
esistono configurazioni che creano collisioni: ad esempio un gioco $(70, 70, 10)$
ha un totale di $3^{4900} \approx 7.8 \cdot 10^{2338}$ stati. Per limitare i
falsi positivi si confrontano il numero di mosse e l'ultimo simbolo giocato.
La cache viene inizializzata all'inizio della partita in un thread dedicato.


\subsection{Principal Variation Search}
Il cuore del giocatore è l'algoritmo PVS (Principal Variation Search). Questo è 
una variante del Minimax basata sull'idea che i nodi dell'albero di decisione 
siano ordinati e che quello in visita correntemente sia gi\`a la scielta migliore.
PVS fa paritre una ricerca sul nodo in questione con una finestra molto piccola 
(alpha e beta quasi uguali). Se questa non conferma quanto assunto allora si fa
partire un'altra ricerca dal nodo in questione, questa volta con una fienstra 
normale.

\subsection{Valutazione Euristica}
Per valutare in modo intelligente gli stati di gioco non terminali viene dato un 
punteggio ad ognuno secondo il seguente criterio: viene calcolato per
ogni riga, colonna, diagonale e antidiagonale quante serie di simboli sono presenti
e si sommano questi valori. In particolare viene scorsa ogni possibile direzione e,
a gruppi di K elementi, viene tenuto conto di quante celle sono libere, quante
quante occupate dal primo giocatore e quante dal secondo. Con questi numeri viene 
calcolato un rapporto e in base a questo si capisce quanto conviene giocare 
quella serie.

\pagebreak
\bibliographystyle{ieeetr}
\bibliography{report}

\end{document}
