\documentclass{article}
\usepackage[utf8]{inputenc}
\usepackage[italian]{babel}
\usepackage[T1]{fontenc}
\usepackage[italian]{varioref}
\usepackage{datetime}
\selectlanguage{italian}
\usepackage[Algoritmo]{algorithm}
\usepackage{algpseudocode}
\usepackage[square,sort,comma,numbers]{natbib}
\usepackage[nottoc,notlot,notlof]{tocbibind}

\hfuzz=100.0pt  % ignore paragraph lengths warnings

\newdate{date}{06}{09}{2021}
\title{\textsc{LittleMan}\\
\large Relazione del progetto per l'insegnamento di Algoritmi e Strutture di Dati}
\author{
  Mattia Girolimetto (0000977478),
  Luca Tagliavini (0000971133)
}
\date{
	Universit\`a di Bologna \\
  \displaydate{date}
}

\begin{document}

\pagenumbering{roman}
\maketitle
\pagebreak
\tableofcontents
\pagebreak

\pagenumbering{arabic}
\section{Problema computazionale}

Lo scopo del progetto \`e quello di implementare un algoritmo efficiente e ottimale
volto alla ricerca delle mosse migliori in un gioco $(m,n,k)$ dove si devono
allineare $k$ simboli in una griglia $m \times n$.

Assumendo non esistano stati invalidi il nodo iniziale (la griglia vuota)
ha $m \times n$ figli, i quali avranno a loro volta $(m \times n)-1$ figli e cos\`i
via fino a profondit\`a $mn$ dove si avranno tutte le foglie. Il numero totale
dei nodi \`e dunque superiormente limitato da $O((m \times n) \cdot ((m \times n)-1)
\cdot \ldots \cdot 1) = O((m \times n)!)$.
Il fattore di diramazione \`e di conseguenza elevato ad ogni stato della partita, il che
impone a un qualunque algoritmo di ricerca per forza bruta la limitazione in
profondit\`a su griglie non banali. L'impossibilit\`a di raggiungere le foglie
richiede una valutazione euristica per poter assegnare un valore ai nodi intermedi
senza esplorarli.

\section{Soluzione adottata}

Il giocatore implementato fa uso di una variante dell'algoritmo \textsc{MiniMax} con potatura
\textsc{AlphaBeta} denominata \textsc{PrincipalVariationSearch}~\cite{negascout}. Questa
consiste in una ricerca limitata in profondit\`a analoga ad \textsc{AlphaBeta}, espandendo 
interamente i nodi pi\`u promettenti e parzialmente quelli restanti. Applicando
una \textsc{IterativeDeepeningSearch}~\cite{id} si possono ordinare i sottoalberi
basandosi sui valori euristici delle ricerche precedenti e si pu\`o raggiungere
la profondit\`a massima nei limiti imposti.

Si nota facilmente che diverse combinazioni di mosse possono portare alla stessa 
situazione di gioco.
Per evitare di analizzarle pi\`u volte, gli stati di gioco vengono quindi mantenuti 
dentro una \emph{tabella delle trasposizioni}. Il valore di quelli non finali
viene stimato da una componente euristica che tiene in considerazione il numero
di serie di ogni giocatore e la relativa lunghezza, favorendo quelle di lunghezza
$k-1$, $k-2$ e $k-3$.

\subsection{Iterative Deepening}

La ricerca della mossa migliore viene gestita da un algoritmo di
\textsc{IterativeDeepeningSearch}~\cite{id}, il quale procede applicando \textsc{MiniMax} 
con profondit\`a sempre maggiore e restituisce l'ultimo risultato trovato prima dello
scadere del tempo. \label{cost:id} Questo non peggiora la complessit\`a asintotica in quanto il
solo costo della ricerca alla massima profondit\`a assorbe quello di tutte
le chiamate precedenti. Tuttavia si ha un aumento delle operazioni totali svolte
dal calcolatore, mitigato dall'utilizzo della \textsc{PrincipalVariationSearch} 
e dalle potature che derivano dall'ordinamento dei nodi.

\subsection{Tabella delle trasposizioni}

La cache fa uso della struttura dati HashMap fornita da Java generando le chiavi 
in modo incrementale tramite la tecnica di \emph{Zobrist}~\cite{zobrist} che restituisce un \verb!long!. 
Ad ognuna di esse sono associate le seguenti informazioni:
\begin{enumerate}
  \item Numero di celle marcate e l'ultimo simbolo giocato
  \item La profondit\`a della ricerca
  \item Il tipo ed il valore della griglia
\end{enumerate}
Poich\`e una tavola di gioco $(m,n)$ pu\`o al pi\`u assumere $3^{m \cdot n}$ stati differenti e
un \verb!long! pu\`o contenere al pi\`u $2^{64} \approx 1.8 \cdot 10^{19}$ valori distinti,
esistono configurazioni che creano collisioni: ad esempio un gioco $(8, 8, 4)$
pu\`o avere fino a $3^{64} \approx 2.3 \cdot 10^{30}$ stati. Per limitare questo
fenomeno ci si assicura che il numero di mosse e l'ultimo simbolo giocato
corrispondano con i valori della griglia attuale.

Per evitare ulteriori collisioni interne alla HashMap, alla fine di ogni turno,
vengono rimosse dalla tabella delle trasposizioni tutte le tabelle di gioco
\emph{inaccessibili}. Una griglia viene definita inaccessibile quando ha un numero di
celle marcate inferiore a quello del tavolo di gioco attuale incrementato di $1$.

\subsection{Principal Variation Search}

L'algoritmo visita in modo ordinato i figli di ogni stato di gioco applicando
limiti $\alpha$-$\beta$ differenti ad ognuno di essi al fine di aumentare il numero di potature.
Si analizza in modo approfondito il nodo pi\`u promettente, denominato \emph{Principal
Variation}, per poi visitare i restanti nodi con una ricerca a finestra \emph{nulla}~\cite{scout}
(dove i due limiti differiscono di $1$). Se questa restituisce un valore compreso
nell'intervallo $\alpha$-$\beta$ originale potrebbe essere d'interesse, perci\`o
viene svolta una ricerca classica.
Nel caso in cui esista una serie $k-1$ per un qualunque giocatore si provvede a
riempire la cella mancante, evitando la ricerca di forza bruta. Viene data la
precedenza alle serie appartenenti al giocatore che deve svolgere la mossa.

\begin{algorithm}[H]
  \caption{\textsc{PrincipalVariationSearch}}
  \label{alg:pvs}
  \begin{algorithmic}[0]
    \Procedure {PVS}{$node$, $color$, $depth$, $\alpha$, $\beta$}
      \State $best\_cell \gets null, \, best\_value \gets -\infty$
      \If{$depth = 0$ \textbf{or} $node$ is a leaf}
        \State \Return $color \cdot \Call{Evaluate}{node}$
      \ElsIf{$best\_cell \gets \Call{FindClosingCell}{node}$ is \textbf{not} $null$}
        \State $child \gets \Call{mark}{node, best\_cell}$
        \State $(best\_value, \_) \gets -\Call{PVS}{child, -color, depth-1, -\beta, -\alpha}$
      \Else
        \For{$child$ \textbf{in} \Call{Sorted}{\textrm{children of} $node$}}
          \If{$child$ is the first}
            \Comment{\`E la \emph{Principal Variation}}
            \State $score \gets -\Call{PVS}{child, -color, depth-1, -\beta, -\alpha}\}$
            \State $\alpha \gets \max\{\alpha, score\}$
          \Else
            \Comment{Proviamo con una \emph{ricerca a finestra nulla}}
            \State $score \gets -\Call{PVS}{child, -color, depth-1, -\alpha-1, -\alpha}\}$
            \If{$\alpha < score < \beta$}
              \Comment{\`E un nodo interessante}
              \State $score \gets -\Call{PVS}{child, -color, depth-1, -\beta, -\alpha}\}$
              \State $\alpha \gets \max\{\alpha, score\}$
            \EndIf
          \EndIf
          \If{$score > value$}
            \State $best\_value \gets score$
            \State $best\_cell \gets $ the last move of $child$
          \EndIf
          \If{$value \geq \beta$}
            \Comment{Potatura $\alpha$-$\beta$}
            \State \textbf{break}
          \EndIf
        \EndFor
      \EndIf
      \State \Return $(best\_value, best\_cell)$
    \EndProcedure
  \end{algorithmic}
\end{algorithm}

Si noti che la logica per la gestione del tempo rimanente e tutti gli accessi
alla tabella delle trasposizioni sono stati omessi per semplificare la lettura.

Nella pratica, se non abbiamo alcuna informazione sull'ordinamento dei figli
essi vengono analizzati in modo casuale, per avere pi\`u possibilt\`a di scoprire
eventuali potature fortuite.

\subsection{Valutazione euristica}

La griglia di gioco mantiene al suo interno una valutazione euristica 
aggiornata progressivamente con l'avanzare della partita (Algoritmo \vref{alg:eval}).
Quando una cella viene marcata si computa il nuovo valore euristico osservando
le righe, colonne e diagonali passanti per essa. Vengono privilegiate le serie consecutive 
di $k-1, k-2$ o $k-3$ elementi, dando un peso inferiore a segmenti pi\`u corti.
La quotazione di un tavolo di gioco \`e il risultato della differenza tra i
valori delle serie dei due giocatori. Inoltre, per favorire le griglie che
assumono valori promettenti con il minor numero di mosse, ogni
valutazione viene divisa per il numero di simboli sulla griglia.

\begin{algorithm}[H]
  \caption{Valutazione eurstica incrementale}
  \label{alg:eval}
  \begin{algorithmic}[0]
    \State $n_{free}, n_1, n_2 \gets 0$
    \Procedure {Eval}{$i$, $j$}
      \State $value \gets 0$
      \State $n_{free}, n_1, n_2 \gets 0$
      \For{$(ii, jj) \in \Call{Row}{j}$}
        \State $value \gets value + \Call{CellValue}{ii, jj, 0, 1}$
      \EndFor
      \State $n_{free}, n_1, n_2 \gets 0$
      \For{$(ii, jj) \in \Call{Column}{i}$}
        \State $value \gets value + \Call{CellValue}{ii, jj, 1, 0}$
      \EndFor
      \State $n_{free}, n_1, n_2 \gets 0$
      \For{$(ii, jj) \in \Call{Diagonal}{i, j}$}
        \State $value \gets value + \Call{CellValue}{ii, jj, 1, 1}$
      \EndFor
      \State $n_{free}, n_1, n_2 \gets 0$
      \For{$(ii, jj) \in \Call{AntiDiagonal}{i, j}$}
        \State $value \gets value + \Call{CellValue}{ii, jj, 1, -1}$
      \EndFor
      \State \Return $value$
    \EndProcedure
    \Procedure {CellValue}{$i$, $j$, $\delta_i$, $\delta_j$}
      \If{$n_{free} + n_1 + n_2 \geq k$} \Comment{Se la serie \`e troppo lunga}
        \State $s \gets B[i - \delta_i \cdot k][j - \delta_j \cdot k]$ \Comment{Stato della prima cella nella serie}
        \State decrement one of $n_{free}, n_1, n_2$ by $1$ based on $s$
      \EndIf
      \State increment one of $n_{free}, n_1, n_2$ by $1$ based on $B[i][j]$
      \Statex
      \If{$n_1 + p_{free} = k$}
        \Comment{Si restituisce la valutazione della serie}
        \State \Return $color \cdot (\Call{LargeSeriesConstant}{n_{free}} + n_1^2)$
      \ElsIf{$n_2 + p_{free} = k$}
        \State \Return $-color \cdot (\Call{LargeSeriesConstant}{n_{free}} + n_2^2)$
      \Else
        \State \Return $0$
      \EndIf
    \EndProcedure
  \end{algorithmic}
\end{algorithm}

\subsection{Altre euristiche considerate}

In questa sezione sono presentati approcci che sono stati considerati,
implementati ma in definitiva scartati in quanto non hanno offerto un
miglioramento apprezzabile allo stile di gioco dell'algoritmo.

\subsubsection{Valutazione euristica degli spazi vuoti}

Nel gioco generalizzato del $(m, n, k)$, specialmente per valori di $k < \max\{m-1,n-1\}$,
una sequenza con celle libere ai lati pu\`o rivelarsi pi\`u vantaggiosa di una
limitata. Ad esempio quando una sequenza \`e del tipo $k-1$ ed ha entrambi i lati liberi
garantisce la vittoria al giocatore. Si pu\`o dunque estendere l'Algoritmo
\vref{alg:eval} affinch\`e sommi un bonus a serie di questo tipo.
Ecco una possibile implementazione che si pu\`o trovare
commentata anche nel codice sorgente:

\begin{algorithm}[H]
  \caption{Valutazione delle serie favorendo spazi liberi adiacenti}
  \label{alg:eval_free}
  \begin{algorithmic}[0]
    \Procedure {CellValue}{$i$, $j$, $\delta_i$, $\delta_j$}
    \State \Comment{La logica rimane invariata fino alla valutazione della serie attuale}
    \State $bonus \gets 0$
    \If{$B[i + \delta_i][j + \delta_j] = free$}
      \State $bonus \gets bonus + \textsc{BonusConstant}$
    \EndIf
    \If{$B[i - \delta_i \cdot k][j - \delta_j \cdot k] = free$}
      \State $bonus \gets bonus + \textsc{BonusConstant}$
    \EndIf
    \Statex
    \If{$n_1 + p_{free} = k$}
      \State \Return $color \cdot (\Call{LargeSeriesConstant}{n_{free}} + n_1^2 + bonus)$
    \ElsIf{$n_2 + p_{free} = k$}
    \State \Return $-color \cdot (\Call{LargeSeriesConstant}{n_{free}} + n_2^2 + bonus)$
    \Else
      \State \Return $0$
    \EndIf
    \EndProcedure
  \end{algorithmic}
\end{algorithm}

Va ricordato che nell'Algoritmo~\ref{alg:eval_free} il controllo della validit\`a delle
posizioni $(i + \delta_i, j + \delta_j)$ e $(i - \delta_i \cdot (k+1), j - \delta_j \cdot (k+1))$
\`e stato omesso per brevit\`a e chiarezza del codice.

% TODO: spiegare perche' non e' stata scelta.

\subsubsection{Analisi limitata ad un intorno}

Per limitare il numero di nodi visitati si \`e pensato di tenere conto solo di
quelli in un raggio di $k$ attorno alle celle marcate. Tuttavia, con l'avanzare
della partita e con l'aumento delle profondit\`a di ricerca, il numero di visite
risparmiate non \`e abbastanza alto da giustificare il costo $\Theta(n)$ del
filtraggio (dove $n$ \`e il numero di celle libere).

\subsubsection{Ricerca quiescente}

Un problema ben noto negli algoritmi di ricerca a profondit\`a limitata \`e
quello dell'\emph{effetto orizzonte}, causato dall'impossibilt\`a 
di visitare l'albero di gioco oltre un certo livello. Ci\`o porta spesso a scegliere 
mosse che si rivelano catastrofiche in pochi. Una soluzione al problema chiamata
\emph{ricerca quiescente} \cite{quiescence} fu per la prima volta proposta da
\citeauthor{quiescence} nel \citeyear{quiescence}.

L'idea \`e quella di esplorare ulteriormente gli stati che appaiono "calmi", o
appunto "quiescenti", poich\`e sono quelli pi\`u proni a degenerare in sconfitte.
Uno stato di gioco viene ritenuto calmo quando il suo valore euristico \`e inferiore
ad una data soglia. Si pu\`o dunque modificare la funzione di valutazione affinch\`e
applichi una \textsc{QuiescensceSearch} ove necessario, come mostrato di seguito:

\begin{algorithm}[H]
  \caption{Ricerca quiescente con struttura \textsc{NegaMax}}
  \label{qs}
  \begin{algorithmic}[0]
    \Procedure{Evalaute}{$board$, $raw$}
      \State $eval \gets $ valutazione euristica
      \If{$raw$ \textbf{or} $|eval| > \textsc{QuiescenceThreshold}$}
        \State \Return $eval$
      \Else
        \State $color \gets 1$
        \If{next moving player of the $board$ is the $enemy$}
          \State $color \gets -1$
        \EndIf
        \State \Return \Call{QuiescenceSearch}{$board$, $color$, \textsc{QuiescenceDepth}}
      \EndIf
    \EndProcedure
    \Statex
    \Procedure{QuiescenceSearch}{$node$, $color$, $depth$}
    \If{$depth = 0$ \textbf{or} node is a leaf}
      \State \Return $color \cdot \Call{Evaluate}{node, true}$
    \EndIf
    \State $best \gets -\infty$
    \For{$child$ \textbf{of} $node$}
      \State $best \gets \max\{best, \Call{QuiescenceSearch}{child, -color, depth-1}\}$
    \EndFor
    \State \Return $-best$
    \EndProcedure
  \end{algorithmic}
\end{algorithm}

La funzione \textsc{Evaluate} viene chiamata dall'algoritmo di ricerca con il valore
dell'argomento $raw$ uguale a $false$.

Nella pratica questa modifica non ha portato a significativi miglioramenti
ma ha caratterizzato l'AI con uno stile di gioco eccesivamente difensivo, il
quale ne ha peggiorato le prestazioni contro automi poco astuti o casuali.
Si pu\`o imputare questo fallimento a due fattori:
\begin{enumerate}
  \item La scarsa precisione della valutazione euristica.
  \item Nei giochi $(m, n, k)$ una qualunque mossa di un giocatore migliora sempre
    il proprio punteggio: di conseguenza i valori delle valutazioni oscillano e
    non convergono mai ad una stima appropriata, in particolar modo nelle fasi
    iniziali della partita. Ci\`o inibisce ampiamente l'effetto della ricerca
    quiescente.
\end{enumerate}

\section{Conclusioni}

\subsection{Costo computazionale}

Per la comprensione del paragrafo che segue defniamo:
\begin{itemize}
  \item $b$ come \emph{fattore di diramazione}, ovvero il numero medio di
    sottoalberi appartenenti ad ogni nodo.
  \item $d$ come la profondit\`a a cui viene svolta la ricerca.
\end{itemize}

La complessit\`a del metodo \verb!selectCell! \`e dominata dal costo di
\textsc{IterativeDeepening} e \textsc{AlphaBeta}. Come gi\`a analizzato in \ref{cost:id}
il primo ha valore trascurabile rispetto al secondo. Il peso della ricerca \`e
compreso tra quello di \textsc{MiniMax} ($O(b^d)$), dove tutti i rami vengono
analizzati, e quello di \textsc{AlphaBeta} \emph{con ordinamento perfetto}
($O(\sqrt{b^d})$), dove solo i nodi strettamente necessari vengono visitati.
Grazie alla combinazione \textsc{IterativeDeepening}-\textsc{PVS} si pu\`o
ottenere un ordinamento ottimo nella quasi totalit\`a dei casi, il che ci avvicina
alla classe di costo $O(\sqrt{b^d})$.

\`E necessario anche quantificare l'apporto significativo delle euristiche,
le quali vengono invocate ogni iterazione. Giocare una mossa ha costo
$O(\max\{m, n\})$ a causa dell'euristica di valutaizone e del controllo
dell'esito della partita. Per la ricerca della cella che riempie una sequenza
$k-1$ il costo \`e pari a $O(m \times n)$ poich\`e vengono controllate tutte le celle libere.
Invece il costo per la generazione delle chiavi e l'accesso alla tabella delle
trasposizioni \`e costante. La complessit\`a totale di queste operazioni
ammonta dunque a $O(\max\{m,n\}) + O(m \times n) = O(m \times n)$.
\subsection{Miglioramenti futuri}

Sono noti in letteratura una serie di algoritmi e valutazioni euristiche utili
nel gioco $(m, n, k)$ o del Go, che potrebbero migliorare le prestazioni
dell'attuale giocatore. Ecco un elenco dei pi\`u significativi:
\begin{enumerate}
  \item Adottare un'euristica per la valutazione pi\`u raffinata. Questo cambiamento
    avrebbe il maggiore impatto: un conteggio delle minacce~\cite{heur}
    analoga a quella proposta da \citeauthor{heur} migliorerebbe significativamente
    la precisione sugli stati pi\`u avanzati del gioco. \label{newheur}
  \item La \textsc{QuiescenceSearch} potrebbe portare ulteriori vantaggi se
    abbinata a una strategia analoga a quella descritta nel punto \vref{newheur}.
  \item Abbadonare la HashMap di Java a favore di una soluzione su misura
    per le necessit\`a del gioco. Ad esempio implementando una semplice cache
    ad indirizzamento diretto, usando i bit centrali della hash di \emph{Zobrist}
    come chiavi e indagando la strategia di rimpiazzamento pi\`u adatta.
  \item Per evitare l'approccio a forza bruta si potrebbe considerare
    la \emph{Ricerca di Monte Carlo}, la quale sfrutta il campionamento casuale
    per stimare la probabilit\`a di vittoria di una data mossa.
  \item \`E stato infine mostrato come altre variazioni di \textsc{AlphaBeta} e
    \textsc{IterativeDeepeningSearch} come MTD(f)~\cite{mtdf} possono portare a
    risultati ancora migliori nel gioco degli scacchi. Riteniamo sia prima
    necessario compiere il passo \vref{newheur} affinch\`e si riveli efficace anche nel nostro ambito.
\end{enumerate}

\pagebreak
\bibliography{report}
\bibliographystyle{IEEEtranN}

\end{document}
