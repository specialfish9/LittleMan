\documentclass{article}
\usepackage[utf8]{inputenc}
\usepackage[italian]{babel}
\usepackage[T1]{fontenc}

\title{Galileo Galilei \\
\large Relazione del progetto per l'insegnamento di Algoritmi e Strutture di Dati}
\author{
  Mattia Girolimetto (0000977478),
  Luca Tagliavini (0000971133)
}
\date{
	Universit\`a di Bologna \\
  \today
}

\begin{document}

\maketitle

\section{Problema computazionale}

Lo scopo del progetto \`e quello di implementare un algoritmo efficiente e ottimale
volto alla ricerca delle mosse migliori in un gioco $(m,n,k)$, dove si devono
allineare $k$ simboli in una griglia $m \times n$.

Si fa uso di una variante dell'algoritmo MiniMax con potatura
AlphaBeta denominata \emph{Principal Variation Search}~\cite{negascout}. Questa
prevede  una ricerca limitata in profondit\`a analoga a minimax, espandendo 
interamente i nodi pi\`u promettenti (denominati \emph{Principal Variation}) e
parzialmente quelli restanti. Applicando un \emph{Iterative Deepening}~\cite{id}
si possono ordinare i sottoalberi basandosi sui valori euristici delle ricerche
precedenti e si pu\`o raggiungere la profondit\`a massima nei limiti imposti.

Diverse combinazioni di mosse possono portare a stati gi\`a analizzati
precedentemente, i quali vengono mantenuti dentro una cache per evitare di
valutarli pi\`u volte. Il valore degli stati di gioco non finali viene stimato
da una componente euristica che tiene in considerazione il numero di serie di
ogni giocatore assieme alla relativa lunghezza e alla vicinanza ad altre celle
libere.

\section{Scelte Progettuali}

\subsection{Select Cell e Iterative Deepening}
Dopo aver aggiornato la copia locale della griglia di gioco si cerca la cella
migliore da giocare sfruttando Iterative Deepening. 
In questo modo si eseguono ricerche MiniMax con profondit\`a sempre maggiore
e si restituisce il risultato pi\`u recente. Questo non
peggiora la complessit\`a in quanto il solo costo asintotico di MiniMax alla
massima profondit\`a assorbe quello di tutte le ricerche precedenti.
Tuttavia aumenta sicuramente il numero di operazioni svolte dal calcolatore.
Per limitare questo danno si ricorre alla Principal Variation Search.

\subsection{Cache}
\subsection{PVS}
\subsection{Valutazione Euristica}

\pagebreak
\bibliographystyle{ieeetr}
\bibliography{report}

\end{document}
